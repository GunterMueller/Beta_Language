"What the heck is BETA?"  This is probably the question, that
many people will ask, when they first hear about BETA: "Why do I
want to use BETA instead of going with the mainstream and using
C++?  What makes BETA different?"

I got fascinated, when I first used BETA.  It was around the
time, that I started to look into object oriented
programming.  I had just learned about C++, and I had gotten
disappointed.  The very first thing which I tried to program in
C++ did not work.  If I had a class hierarchie of objects, like
class "lightings", subclass "bulbs", subclass "halogen bulbs", I
wanted to also have a hierarchie of lists of these objects.  That
would mean, that a list of bulbs could be passed to a function,
that needs a list of lightings as argument.  I had to learn, that
this is in principal not possible with C++.  I began to ask
myself: "Why do I want to use a language, which does not even
solve the easiest problem, that I could think of to test it
with?"

Around the same time, I heard about BETA by incident.  And I
found, that one of the examples delivered with the Mjolner
compiler described a BETA solution of my problem.  And shortly
later, I had to learn, that BETA is extremely simple to learn.
To learn C++, you have to read and understand a big book.  To
learn BETA, a short tutorial is enough.

I started to get fascinated.  I don't have to study for long,
and can already use BETA.  And the more I used it, the more
I found out, how versatile it is.  There is only one major
construction element, which is the \tq{pattern}.  Patterns
can do, what functions and classes do in other languages.
And by cleverly combining patterns, patterns can do, what
templates, methods, environments or systems do in other
languages.  There is just this one universal building block,
which does it all for me.  I like this very much.

I hope, that as many people as possible will be able to join
me in having fun with BETA.  Therefore, I have choosen to
write a BETA compiler as freeware, and make it available as
wide as possible.  This document both describes how to use
the compiler, and how the compiler was programmed.

Have fun,
\vspace{3cm}

Kai Petzke
