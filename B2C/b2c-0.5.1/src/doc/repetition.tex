\section{Repetitions}
The repetitions defined by BETA are more than just plain arrays.
Rather, many of the operations on repetitions resemble the
behaviour of specialized patterns respective objects:
\begin{itemize}
\item Certain methods (namely \cq{new}, \cq{extend} and
    \cq{range}) operate on repetitions pretty much like "normal"
    methods on conventional objects.
\item Repetitions have an exit list, that will exit the values of
    all the elements in the repetition.
\item Repetitions have an enter list, that can accept the elements
    exited from another repetition.
\item The size specifier in a repetition declaration may be any
    valid BETA expression.  Evaluating that expression may have
    side effects on other BETA objects.  Normal object generation
    does not have such side effects.

    In other words: the size specifier adds functional aspects
    to a repetition, like the do part adds functional aspects
    to a pattern.
\end{itemize}

This compiler actually implements BETA repetitions by turning them
into patterns.  Directly after scanning the declaration of a
repetition:
\begin{quote}\tt \tq{name}: [\tq{expression}] @\tq{type}
\end{quote}
it is turned into the following:
\begin{quote}\verb+repetition:+\\
   \verb+  (#+\\
   \verb+     size: @integer;+\\
   \verb+     elsize: @integer;+\\
   \verb+     range: (# exit size #);+\\
   \verb+     new: (# enter size do (* +{\it create repetition}\verb+ *) #);+\\
   \verb+     extend:+\\
   \verb+       (# by: @integer enter by do (*+
		    {\it extend it\/} \verb+*) #);+\\
   \verb+  do INNER+\\
   \verb+  #);+\\
   \verb++\\
   \tq{name}\verb+: @repetition+\\
   \verb+  (# ptr: ^+\tq{type}\\
   \verb+  do +\tq{expression}\verb+->size+\\
   \verb+  #);+
\end{quote}

This translation arises a few problems, which have to be considered
carefully:
\begin{itemize}
\item The \cq{repetition} superpattern is defined in the main
    \cq{betaenv} environment.  It may only be used internally by
    the compiler as a superpattern for the specific repetition
    patterns as shown above.
\item Direct access to the \cq{size} and \cq{elsize} fields from
    outside of the repetition's own methods and do part is not
    legal.
\item The \tq{expression} and the \tq{type} of the original
    repetition declaration are moved to a different level.  This
    is not a problem except for resolving variable names.  For
    example, when \tq{type} is \cq{new}, it should refer to a
    \cq{new} pattern outside of the repetition, not the
    repetition's own \cq{new} method.
\end{itemize}

Translating repetitions into patterns also causes a few small
performance degradations at run time:
\begin{itemize}
\item Repetitions inherit the overhead of all BETA objects, namely
    the pointer to an array of run time type information, and the
    pointer(s) to the enclosing object(s).
\item The repetition's initial size is always calculated by
    calling a separate do part, even, if this expression is
    constant.
\item Moving the repetition's \tq{type} and \tq{expression} one
    level deeper may add one level of indirection to variable
    lookup.
\end{itemize}
Some of these disadvantages will go away, once the compiler learns
to actually do BETA specific optimizations to the code.  At the
moment, they are not considered to be a problem.
