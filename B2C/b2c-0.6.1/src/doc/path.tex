\section{Example}
When all the elements described above are used, an attribute
denotation can get quite long, as in the following example:
\begin{quote}\begin{verbatim}this(program).obj.geom.corner[1].x
\end{verbatim}\end{quote}
For this denotation to be grammatically correct, several
definitions like the following ones will have to be in your
program:
\begin{quote}\begin{verbatim}program: <<SLOT Program: ObjectDescriptor>>;
--- Program: ObjectDescriptor ---
(#
   window: (# geom: @geometry #);
   obj: ^window;
   geometry:
     (#
        corner: [2] ^point;
        point: (# x,y: @integer #)
     #);

   movex:
     (# x: @integer
     enter x
     do
        obj.geom.pt[1].x+x->obj.geom.pt[1].x;
        obj.geom.pt[2].x+x->obj.geom.pt[2].x;
     #);
do
   (* initialization not shown here *)
   5->movex;
#);
\end{verbatim}\end{quote}
The denotation \cqq{this(program).obj.geom.corner[1].x} is used
in the DO part of the \cq{movex} pattern.  The first part
(\cqq{this(program).}) has been implicitely added by the
compiler, because it does not find any attribute named \cq{obj}
in the \cq{movex} pattern and then continues looking for it in
the enclosing patterns.

\section{Paths}
I think of these long attribute denotations as paths.  On
each step, a specialised denotation will tell, what to do
next.  So, the above path reads as follows:
\begin{itemize}
\item Go back from the \cq{movex} object to the \cq{program}
    object.
\item Find the \cq{obj} object within \cq{program}.
\item Find the \cq{geom} object within the object, that
    is referenced by \cq{obj}.
\item Find the \cq{corner} object within \cq{geom}.
\item Denote the first element of the \cq{corner} repetition.
    This results in an object of type \cq{point}.
\item Find the \cq{x} element in this \cq{point} object.
\end{itemize}

\section{Joining Paths}
A common operation during the compilation of a BETA program is to
join two paths into one.  Typically this means to fit the head of
path 2 to the tail of path 1:
\begin{quote}\begin{verbatim}(#
   point: @
     (#
        x,y: @integer;
        setx: @(# xx: @integer enter xx do xx->x #);
     #);
do
   1->point.setx
#)
\end{verbatim}\end{quote}
The assignment \cqq{1->point.setx} requires to set that variable
to 1, which is denoted by \cq{xx} in the \cq{point.setx}
object.  The next step then is to call the do part of the
\cq{point.setx} object.  So the compiler will translate the
above example into the following code:
\begin{quote}\begin{verbatim}(#
   point: @
     (#
        x,y: @integer;
        setx: @(# xx: @integer enter xx do xx->x #);
     #);
do
   1->point.setx.xx;
   point.setx;
#)
\end{verbatim}\end{quote}
This example shows, that the enter and exit lists of an object
are processed completely at the side of the caller of a given
function.  The reasons for this are explained in section
\ref{call-do-part}.  In our example, this requires to write the
value 1 into the \cq{xx} variable of the \cq{point.setx} pattern.
In the final assignment, these two attribute denotations will be
joined together, resulting in the term \cq{point.setx.xx}.

This is by far not the only possible situation.  One could change
the above example as follows:
\begin{quote}\begin{verbatim}(#
   point: @(# x,y: @integer; setx: @(# enter x #) #);
do
   1->point.setx;
#)
\end{verbatim}\end{quote}
If we did the same joining of the two attribute denotations as
described above, we'd get \cq{point.setx.x} as result.  But that
can't be right.  The \cq{setx} object does not have an \cq{x}
attribute.

There is something, that we forgot: the \cq{x} in the enter
section of \cq{setx} is prepended with an implicit
\cq{this(point)} by the compiler.  So the correct joint attribute
denotation reads: \cq{point.setx.this(point).x}.  This can be
shortened into \cq{point.x}, as the parts \cq{setx} and
\cq{this(point)} cancel each other out.

\section{Joining Rules}
Yet to be written.
