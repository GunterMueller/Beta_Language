\section{Special Object Descriptors}
As any other language, BETA needs a set of basic types and basic
functions respective methods on which it can operate.  An example
of a basic type is \cq{Integer}, a basic method is the
\cq{extend} method of a repetition.  A common feature of these
basic types and functions is, that they cannot be represented
within the language itself.

Most languages therefore define a set of keywords for these types
and functions, like the type \cq{int} or the function
\cq{sizeof()} in C.  This is not true in BETA.  There are no
keywords reserved for the basic types.  It is therefor possible
to redefine names like \cq{Integer} in your own patterns.

However, the compiler must still be able to recognise the basic
types.  Therefore, certain pattern names are given special
meaning --- but only in the outermost fragment, that is typically
called \cq{betaenv}.  The list of special names is:
\begin{quote}
	Boolean \\
	Char \\
	Integer \\
	Real \\
	Shortint \\[2ex]
	Data \\
	External \\
	Object \\
	Repetition \\
	Text
\end{quote}

Most special descriptors are accounted for in the fragment
\ffq{PatternDeclCheck0}{DoPart}{attributebody.bet}.  The single
exception is the \cq{Object} pattern.  Because it is the
superpattern of all other patterns, it has to be checked before
any other pattern is processed.  This special treatment of the
name \cq{Object} is performed in
\frq{PatternDeclCheckName}{DoPart}.

\subsection{The basic types}
The five patterns \cq{Boolean}, \cq{Char}, \cq{Integer},
\cq{Real} and \cq{Shortint} all represent a storage cell, which
holds a value.  To generate an object from one of those patterns
means to allocate an appropriate storage cell.  To enter an value
into that object means to put that value into the storage cell.
To exit a value from the object means to retrieve the value
stored in that cell.  In neither case, a DO-part is executed.

To ensure type safety without sacrificing efficiency, all
references to objects of these basic types must be static
references.  That means, that these objects are well defined
within the enclosing pattern, and only within that pattern.
Objects of the basic types don't have a pointer to the pattern
information or to the enclosing object.

The backdraw is, that the basic types cannot be extended in any
way.  No subclassing, no further attributes, no DO-Part is
possible.  To partly overcome that shortage, the basic library
\fq{betaenv.bet} defines full pattern versions
\cq{IntegerObject}, \cq{CharObject}, \ldots\ of the basic types.

\subsection{The \cq{Data} pattern}

\subsection{The \cq{External} interface}

\subsection{The \cq{Text} type}

\subsection{The main superpattern: \cq{Object}}
