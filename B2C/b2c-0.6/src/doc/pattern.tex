\section{Pattern structure}
All standard BETA objects have to carry around a certain amount
of run time type information.  Rather then including this
information into each object, most of it is collected into a
static structure at compile time.  The objects themselves
therefore only need to carry a pointer to that structure.  This
pointer is located at a fixed position at the beginning of
an object's structure; currently this pointer is the first
element of an object.

\subsection{Pattern relationships}
BETA defines certain relationships between patterns, respective
the objects generated from them.  Any given pattern $P$ has a
superpattern $S(P)$.  The single exception is the \cq{Object}
pattern, which has no superpattern.  However, \cq{Object} is the
implicit superpattern of all patterns, whose supperpatterns were
not specified.

Every pattern $P$ also has a statically enclosing pattern $E(P)$.
$E(P)$ is the pattern, whose attribute section holds the
declaration of $P$.  There are two exceptions: the enclosing
pattern of \cq{Object} may be any pattern or empty.  And because
\cq{Betaenv} always is the outermost pattern, its enclosing
pattern is not defined, either.

One might assume, that a pattern $P$, and its superpattern $S(P)$
have the same enclosing patterns, so that $E(P)$ = $E(S(P))$.  But
this is in general not true in BETA.  A common example are virtual
patterns:
\begin{quote}\begin{verbatim}env: (# virt1:< (# ... #) #);
myenv: (# virt1::< (# ... #) #);
\end{verbatim}\end{quote}
Clearly, the \cq{virt1} pattern of \cq{env} has \cq{env} as
enclosing pattern, while the \cq{virt1} pattern of \cq{myenv}
has \cq{myenv} as enclosing pattern.  However, in this special
example, the equation: $E(S(\cq{myenv.virt1})) =
S(E(\cq{myenv.virt1}))$ holds true.  These and some other
relationships will be detected by the compiler, and used for
various optimizations.  But the general situation in BETA
is, that any pattern $P$ has
a set of enclosing patterns $E(P)$, $E(S(P))$, $E(S(S(P)))$,
\ldots, $E(S^n(P))$, where $S^n(P)$ is the n-th superpattern
of $P$.

\subsection{Definition of \cq{struct pattern}}
All required run time type information is stored in a special
pattern structure.  Objects only hold a pointer to that structure
instead of reproducing the run time type information over and
over again.

Like object size may vary, the size of the run time type
information is not fixed either, but depends on the details of
the pattern, like the number of virtual methods used.  However,
the first few fields of the structure are well defined for (almost) all
patterns.  This common header is defined as \cq{struct pattern}
in the header file \fq{beta.h}:
\beq\begin{verbatim}/* standard structure for storing patterns */
struct pattern {
    const struct pattern *super;
    const struct pattern *outer;
    int len;
    int outerindx;
    int (*function)(void *const th);
    void *(*newf)(void *const th);
};
\end{verbatim}\eeq

The meanings of the fields are as follows:
\begin{itemize}
\item \cq{super} is a pointer to the superpattern of this pattern,
    introduced as $S(P)$ in the notation above.
\item \cq{outer} refers to the structure of the enclosing pattern $E(P)$.
\item \cq{len} is the length of the objects, that can be generated
    from this pattern, in bytes.
\item \cq{outerindx} is an index into the objects generated from
    this pattern.  It denotes the location, where the pointer to
    the enclosing object $e(o)$ of that object is stored.
\item \cq{function} is the first slot of the call table for the
    various DO parts of this patterns and its superpatterns.  In
    the case of empty or only trivial DO parts, the value zero
    will be stored here; otherwise this cell will contain the
    pointer to a function, which executes the complete DO part
    of this pattern and its superpatterns.  This function requires
    one argument, which must be a pointer to an object generated
    from this pattern.

    After successfull completion of the directives in a DO part,
    this function will return the value zero.  If an leave or
    restart imperative occurred, which could not be catched in
    the DO part itself, a handle for that imperative will be
    returned.  It is the responsibility of the caller to test
    the return value and to process the leave/restart handles.
\item \cq{newf} is a pointer to the object generator function.
    As even the most trivial objects have an object generator
    function, this pointer is never zero.  Object creator
    functions take exactly one argument, which is a pointer
    to the enclosing object of the newly created object.
    This argument must be nonzero, except, if the creator of
    the \cq{Betaenv} object is called.  In either case, this
    function will return a pointer to the newly generated
    object.
\end{itemize}

\section{Objects}
Storage of BETA objects is different from that of the patterns.
While a pattern and its superpattern are stored in different
structures, an object and its ``superobjects'' are always unified
into one structure.  The first element of an object is always a
pointer to the pattern structure for that object, then follow the
data fields and pointers, that were declared in the attributes
sections of the relevant pattern and its superpattern(s).  The
fields declared in the outermost superpattern come first, then
those declared in the second outermost superpattern, and so on.
So declaring a subpattern of a pattern will extend the object
structure at the end, while the rest of it is unchanged.

To keep track of the size of an object, the pattern structure
holds the field \cq{len}, which stores the length of the
complete object structure of an object generated from that
pattern.

\subsection{Relations between enclosing objects, and reuse of pointers
    to them}
While an object and its ``superobjects'' are always inlined into
one structure, an object and its enclosing objects are not.
Rather, any object holds a set of pointers $e_1$, \ldots\ $e_n$
that point to the enclosing objects.  The compiler tries to
minimize the number of these pointers to enclosing objects
by re-using them in a few cases, like when two enclosing
objects are the same: $e_k = e_i$, or superpatterns of each
other: $e_k = s^j(e_i)$.  In both cases, $e_k$ can be omitted.

Inside the C structure of the object, these pointers are
named \cq{encl1}, \cq{encl2} and so on.

In general, for any item in the superpattern list of a pattern, a
different pointer to an enclosing object is needed.  However,
these pointers often denote the same object, and it is wise to
apply a few rules to reduce the number of pointers needed:
\begin{itemize}
\item Even though the enclosing pattern of the \cq{Object}
    pattern is undefined, objects of type \cq{Object} still need
    a pointer $e_1$ to an enclosing object.  Otherwise, the
    system would not be able to perform qualification checks at
    run time.

    But because the \cq{Object} pattern does not define any
    meaning to the enclosing pointer $e_1$, any direct subclass
    of \cq{Object} can recycle that pointer $e_1$ for their
    own purposes, instead of defining a new $e_2$.

\item If a pattern and a subpattern are defined inside the same
    attribute section, they have the same enclosing pattern.
    Objects generated from these also have the same enclosing
    object, so the enclosing object pointer can be reused.

\item Another common situation is, that the enclosing pattern of
    the superpattern of $P$ is a superpattern of the enclosing
    pattern of P: $E(S(P)) = S^n(E(P))$, with $n \ge 1$.
    This is typically the case with virtuals:
    \beq\begin{verbatim}P: (# V:< (# ... #) ... #);
Q: P (# ... #);
R: Q (# V::< (# ... #) ... #);
\end{verbatim}\eeq
\end{itemize}
After these rules have been applied typically only one or two
pointers are left.  But even these are not independant.  If you
have the enclosing object $e(o)$ of an object $o$, the enclosing
objects of the superpattern parts of $o$ (like $e(s(o))$,
$e(s(s(o)))$ and so on) can be determined by using a
recursive scheme:
\begin{itemize}
\item If $o$ is of type \cq{Object}, we are done, as \cq{Object}
    has no superpattern, thus no $e(s(o))$ can be determined.
\item If $o$ is generated from a pattern, that is a direct
    subpattern of \cq{Object}, $s(o)$ is of type \cq{Object}.  So
    any object may serve as $e(s(o))$, and we set $e(s(o)) =
    e(o)$, by applying the first pointer reuse rule above.
\item In any other case, $o$ must have been generated from a
    pattern $P$, that has an explicit superpattern $S(P)$.
    $S(P)$ is denoted using a \tq{prefix} in BETA:
    \beq\tt P: \tq{Prefix} (\# ... \#)
    \eeq
    The \tq{Prefix} can be defined in one of three ways:
    \begin{enumerate}
    \item If \tq{Prefix} is a simple pattern name, it must be
	defined in any of the enclosing patterns of $P(o)$, like
	$E(P(o))$, $E(E(P(o)))$, and so on.  Otherwise, that
	pattern name would be an illegal name reference.  So
	$E(\mbox{\tq{Prefix}}(P(o))) = E(S(P(o)))$ must be equal
	to $E^n(P(o))$, with $n \ge 1$.  To get the corresponding
	enclosing object, simply leave out the pattern lookup:
	$e(s(o)) = e^n(o)$.
    \item If \tq{Prefix} is of the form
	\tq{attribute-denotation}.\tq{pattern-name}, things are
	even easier.  \tq{attribute-denotation} must denote an
	object, and this object is the enclosing object of the
	prefix, thus that of the superpattern.
    \item In case of a virtual binding, the prefix is given as
        the preceding virtual declaration or virtual binding,
	which must be present in a superpattern of the enclosing
	pattern.  So: $e(s(o)) = s^n(e(o))$ for some $n \ge 1$.
    \end{enumerate}
\item Repeat the above scheme with $s(o)$ instead of $o$ to
    determine the enclosing object $e(s(s(o)))$.
\end{itemize}
Summarization: only $e(o)$ is required to generate the full set
of pointers to enclosing objects $e(s^n(o))$, with $n \ge 0$.
Thus the term "the enclosing object" will be used from now on to
refer to $e(o)$.

\subsection{Full type qualification}
From what was said before, we can conclude, that two pieces of
information are required to fully qualify the type of an object
in BETA:
\begin{itemize}
\item A pointer to the pattern $P(o)$.
\item A pointer to the enclosing object $e(o)$.
\end{itemize}

\subsection{Static and dynamic references}
BETA objects may contain static and dynamic references.  There are
two possible methods of implementation:
\begin{itemize}
\item Store a pointer to the object denoted by this reference in
    the object structure.
\item Inline the denoted object into the object structure.
\end{itemize}
Inlining in general is the more performant choice.  However,
inlining is not possible in quite a few cases:
\begin{itemize}
\item Dynamic references can never be inlined, because the denoted
    object may be changed by a reference assignment.
\item Static references to virtual patterns cannot be inlined,
    because in a subclass, the virtual might be extended, and
    would then no longer fit into the place, that has been
    reserved for it in the inlining process.
\item Static references to object descriptor slots cannot be
    inlined, either.
\end{itemize}
If none of these rules apply, inlining is performed.

\subsection{Generating objects}
When a BETA object is generated, the following actions have to be
performed:
\begin{itemize}
\item Allocate sufficient memory.
\item Clear that memory, so that all numbers are initialized to
    zero, all booleans are set to false and all references point
    to NONE.
\item Set the pointer to the pattern structure.
\item Set the pointer(s) to the enclosing object(s).
\item Initialize all static references of this object.
\end{itemize}
Especially the last two actions (setting the pointers to the
enclosing objects and initialising all static references) are
rather object specific.  It is hard to operate them from a table.
Therefore, the compiler emits a specific object generator
function for each pattern.

This function takes just one argument, which is the pointer to
the enclosing object.  All other information, like the pointer to
the pattern structure, the lookup of the complete list of
enclosing objects and the initialization of the static references
are hardcoded into the function.

In the last step, when static references are initialised, an
object generator function may (and sometimes has to, especially,
when virtuals are involved) call other object generator
functions.  However, when  object generator functions call other
object generator functions, they will never loop, as recursive
static references, where an object contains itself, are forbidden
in BETA for obvious reasons.

The exception to this rule is caused by repetitions: the
definition of a repetition contains a general BETA expression,
which calculates the initial size of the repetition, and this
expression may well cause recursive calls of the object creator
function.

Generating objects in general has very few side effects.  In most
cases, despite the allocation of memory, no other global data is
changed.  The exception are cases, where the \tq{prefix} of a
pattern or a static reference is a computed remote, a reference
lookup or similiar.

\subsection{Generating objects from virtual patterns}
BETA's virtual pattern can be used in two different ways:
\begin{itemize}
\item Their type can be queried, either explicitely with the
    operator \cq{\#\#} or implicitely in a run time qualification
    check.  In this case, the virtual pattern is used as a virtual
    type.
\item They can be used to instantiate BETA objects with the
    operator \cq{\&}, in a declaration or in an inserted item.
    After object generation has been completed, the DO part may
    be called.  This means to use the virtual pattern like a
    virtual function.
\end{itemize}
These operations are not independant of each other.  If we have
queried the full type of a virtual, we can always use the type
information to directly generate an object from it.  And if an
object has been created, the type can be queried from it, too.
However, object generation may have side effects, so it is not a
good idea to perform it, unless it is required explicitely.

The full type qualification of a virtual $v$ consists of the
pattern $P(v)$ and the enclosing object $e(v)$.  While the
pattern information could be retrieved by a simple indirect table
lookup into the environment of the virtual, the enclosing object
$e(v)$ can be a somewhat arbitrary function of that environment.
Therefore, we need a C function to do the virtual type lookup.
It takes two parameters on input: a pointer to the object, in
which the the virtual is to be looked up, and a pointer to
storage space for the full type qualification.  As a convenience
and for efficiency, if the later pointer is NULL, an object with
the type of the virtual will be generated and a pointer to the
object is returned.

\subsection{Destroying objects}
There is no method or function in BETA, which can be used for
destroying an object.  Rather, the compiler and runtime libraries
are responsible for keeping track which objects are still in use.

In special cases, the compiler might already be able to determine
at compile time, that a given object has a limited lifetime, and
allocate it in a space, where it can be easily deallocated, like
on the stack.  In general, though, the compiler cannot know, how
long an object will be around.  It is therefore necessary at run
time to examine, which objects are no longer used, and reclaim
the storage allocated to them.  This process is called
\tq{garbage collection}.

An experimental version of a garbage collector has been
implemented by Jan Okrouhly.

\section{Calling an object's DO-part}
\label{call-do-part}
In procedural languages like C, calling a function means, that
this function will first allocate some local storage on the stack,
then execute some code, deallocate the storage, and return.  This
is not true for BETA.  Generating the storage for an object, and
calling a DO--Part are very well distinct processes.

In procedural languages, passing parameters between a calling and
a called function is a shared process: the calling function will
store the values of the parameters in a reserved region (like
processor registers or the stack), and the called function will
pick them up there - and later put the return value in the same
or another reserved region.  This is not true for BETA.  For
example, the caller may only provide an incomplete set of
parameters to the called function, as in the following example:
\begin{quote}\begin{verbatim}env: (#
   f1: (# A: @integer enter A do A->putint; INNER #);
   f2: f1
     (# B: @integer
     enter B
     do ' '->put; B->putint; INNER
     #);
   X: ^f1
do
   &f2[]->X[];
   1->X
#);
\end{verbatim}\end{quote}
Here, the call \cq{1->X} will execute the joint do part of
\cq{f1} and \cq{f2}, but processes only the ENTER part of
\cq{f1}.  Therefore, the ENTER and EXIT directives are completely
processed in the calling function, not in the called function.
The general calling sequence for the Do-Part of a pattern
therefore reads as follows:
\begin{enumerate}
\item Generate the object, that shall be called.
\item Process the ENTER part.
\item Call the code, that processes the imperatives of the do part.
\item Process the EXIT part.
\end{enumerate}
Any of these steps is optional, depending on the exact form of the
expressions in the call.

\subsection{Executing the INNER imperative}
The INNER imperative is a placeholder, where the imperatives of a
subpattern will later be inserted.  There are two options on how
to translate the nested do parts of a pattern and its superpattern
to C functions:
\begin{itemize}
\item The compiler recursively interprets all INNER imperatives,
    and generates code, which executes the complete do part of a
    pattern --- from the outermost superpattern to the pattern
    itself.  However, this scheme may result in large executable
    programs, as the object code of the do-part of a given
    pattern has to be repeated in all subpatterns of that
    pattern.

    To execute the do part of a given object, the specific
    function for that object's type has to be called.
\item The compiler replaces all INNER imperatives with an
    indirect function call of the appropriate INNER function.  No
    code repetition is necessary in this case, but the indirect
    function call causes delay, and the indirect function call
    table increases the size of the pattern structure.

    To execute a DO part, the DO part function of the
    outermost super--object has to be called; this function
    will automatically call all INNER functions via the
    indirect call tables.
\end{itemize}

Currently, the indirect call scheme has been implemented.  In a
later version of the compiler, it should also support the
inlining of INNER parts.  Depending on some heuristics, like the
size of the resulting function, the compiler should decide,
which scheme to use.

\subsection{Handling of empty and trivial DO parts}
Many BETA objects have an empty DO part.  No function call is
generated by the compiler for them.

Other BETA objects, namely abstract superpatterns, have trivial
DO parts, that contain nothing but one INNER imperative.  The
most prominent such pattern is the \cq{Object} pattern.  For
these patterns, no function is generated, either.  The call table
slot is left open, until a subpattern defines a non-trivial DO
part.

In some cases, where the DO part of a dynamic reference is to be
executed, the compiler may not know at compile time, which
function to call or whether to call a function at all.  In that
case, an indirection through the first slot of the call table is
performed, which is very similiar to the lookup of INNER.

\subsection{LEAVE and RESTART}
There are two more BETA imperatives, that affect the flow of
execution in non-standard ways: LEAVE and RESTART.  In many
cases, they can be handled locally and translated into a goto
statement in C, like where a loop is restarted or left.

However, LEAVE and RESTART can also be used to return control
to a specific function on the call stack.  In these cases, two
informations have to be provided:
\begin{itemize}
\item A pointer to the object, whose DO part should catch the
    LEAVE or RESTART.  That object's DO part must be on the
    calling stack --- or a run time error will occur.
\item A unique identifier, which denotes the label inside the DO
    part, which shall be restarted or left.
\end{itemize}
These two informations are combined into a leave/restart handle.
The identifiers are implemented as pointers to predefined
character strings.  That way, we easily ensure that no identifier
value is repeated.  Also, the character string can hold details
about the actual operation being performed, which make the error
message more verbose in case, that the complete stuck is unwound
in a LEAVE or RESTART.

The current BETA syntax is quite strict, in that a leave or
restart statement may only occur in the DO part of a given
pattern.  Therefore, the following is illegal:
\begin{quote}\begin{verbatim}interpreter:
  (#
     abort: (# do leave interpreter #);
     syntax_error:
       (# do 'Syntax error!'->putline; abort #);
     read_token:
       (#
       do
          ...
          (if some_error_condition then syntax_error if)
          ...
       #);
  do
     read_token;
  #)
\end{verbatim}\end{quote}
However, the following will work:
\begin{quote}\begin{verbatim}interpreter:
  (#
     abort: (# do throw_it #);
     throw_it: ^Object;
     syntax_error:
       (# do 'Syntax error!'->putline; abort #);
     read_token:
       (#
       do
          ...
          (if some_error_condition then syntax_error if)
          ...
       #);
  do
     l1:
       (#
       do
          &(# do leave l1 #)[]->throw_it[];
          read_token;
       #);
  #)
\end{verbatim}\end{quote}
The advantage of the strict syntax is, that we will know at
compile time, which LEAVE/RESTART imperatives may occur, and
which will not.  So we can restrict the number of handles to the
required minimum.  In the first example, the \cq{abort} method
might be defined in a different fragment in a different source
file, and the run time system might be quite surprised to receive
a handle for the \cq{leave interpreter} imperative.
