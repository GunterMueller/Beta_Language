\section{Computed Remotes}
\label{comprem}
A special feature of the BETA language is the \em Computed Remote\/\em\
attribute denotation.  It has the syntax:
\begin{quote}{\tt (}{\it evaluation\/}{\tt ).}{\it attribute}
\end{quote}
There are two restrictions on {\it evaluation\/}:
\begin{itemize}
\item It must compute to something, that has an exit list of length one.
\item The exited value must be an object reference.  The referenced
    \tq{object} becomes the base for the lookup of \tq{attribute}.
\end{itemize}
To correctly process the lookup of \tq{attribute}, we must know
the type of \tq{object}.  That in principle requires to analyze
\tq{evaluation}.  However, in the compiler, all attribute
denotations including the computed remotes are processed before
any evaluations are checked.  Therefore, the normal routines for
the reduction of evaluations cannot be used to determine the
requested type.  Rather, an individual routine has to be written.
As said before, the to be analyzed \tq{evaluation} must have
exactly one element on its exit list, and that element must
reference an object.  This leads to the following alternatives:
\begin{enumerate}
\item \tq{evaluation} is an object reference, that is either
    \cqq{\tq{o}[]} with some object \tq{o} or \cqq{\&\tq{p}[]}
    with some pattern \tq{p}, where \tq{p} can be both specified
    as a name or as an object descriptor.  No further processing
    is necessary in this case, the type of the evaluation is
    either \tq{p} or the object descriptor of \tq{o}.
\item \tq{evaluation} is an object evaluation, as in
    \cqq{\tq{o}}, \cqq{\&\tq{p}} or \cqq{\tq{p}}.  Then we will
    have to look up the exit list of \tq{o} respective \tq{p},
    replace \tq{evaluation} with it, and go back to step 1 above.
\item There is an assignment \cqq{\tq{exp1}->\tq{exp2}}.
    Use \tq{exp2} as new \tq{evaluation} and start over.
\end{enumerate}
These rules may be too strict.  In particular an evaluation like
\cqq{(\tq{A},\tq{B})}, where \tq{A} exits zero values and \tq{B}
exactly one value would in principal be legal for a computed
remote, but is currently rejected.  On the other hand, the only
reason, why one would want to put \tq{A} into that evaluation is
because of its side effects.  Using a computed remote in an
attribute denotation to force such side effects usually is just
bad coding style, making it a good thing for the compiler to
reject it.
