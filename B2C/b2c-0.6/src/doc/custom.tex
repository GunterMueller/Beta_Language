% This file defines a few macros for TeX ...  Change them, if you want
% to change the appearence of this document.

% file quote: put a file name in quotes, or print it emphasized, or both:
\newcommand{\fq}[1]{{\em #1\/\em}}

% fragment quote: how do you want fragment names to appear in this document:
\newcommand{\frq}[2]{--\,-- #1: #2 --\,--}

% fragment file quote: specify a Fragment and a file:
\newcommand{\ffq}[3]{--\,-- #1: #2 --\,-- in file \fq{#3}}

% command quote: put a command in a typewriter type font:
\newcommand{\cq}[1]{{\tt #1}}

% command quote quoted: like \cq, but add quotes around the command:
\newcommand{\cqq}[1]{``{\tt #1}''}

% term quote: emphasize an important word or term:
\newcommand{\tq}[1]{{\em #1\/\em}}

% begin/end example quote: use this to indent pieces of code or examples:
\newcommand{\beq}{\begin{quote}}
\newcommand{\eeq}{\end{quote}}


% If you use the cm-fonts, use the following definitions:
%\newcommand{\ttlq}{\`{ }}
%\newcommand{\ttrq}{\'{ }}
%\newcommand{\tthoch}{\^{ }}

% The following defs are ok for the users of Postscript fonts:
\newcommand{\ttlq}{`}
\newcommand{\ttrq}{'}
\newcommand{\tthoch}{\^{ }}

% a list of new commands:
% \backslash prints \
% \leftbrace prints {
% \rightbrace prints }
\begingroup \catcode `|=0 \catcode `[= 1
\catcode`]=2 \catcode `\{=12 \catcode `\}=12
\catcode`\\=12
|gdef|backslash[\]|gdef|leftbrace[{]|gdef|rightbrace[}]
|endgroup

% \leftbracket prints [
% \rightbracket prints ]
% \hoch prints ^
% \underscore prints _
% \dollar prints $
\begingroup \catcode `[= 12 \catcode`]=12 \catcode`^=12
\catcode`_=12
\catcode`$=12
\catcode`"=12
\gdef\leftbracket{[}\gdef\rightbracket{]}\gdef\hoch{^}\gdef\underscore{_}
\gdef\dollar{$}\gdef\dquote{"}
\endgroup

% actually, if a TeX guru sees this file, I would be happy, if he told
% me, how to put these commands into boxes, so that these commands are
% even safe in those cases, that the TeX-Interpreter is called recursively.
